\documentclass[10pt,a4paper]{article}

\input{AEDmacros}
\usepackage{caratula} % Version modificada para usar las macros de algo1 de ~> https://github.com/bcardiff/dc-tex


\titulo{Trabajo Práctico 1}
\subtitulo{Especificación y WP}

\fecha{\today}

\materia{Algoritmos y Estructuras de Datos I}
\grupo{Grupo "gliptodonte24"}

\integrante{Maydana, Daniel}{205/22}{danimaydana9@gmail.com}
\integrante{Lozada, Jack}{1142/22}{nothingbutjack2200@gmail.com}
\integrante{Cian, Andrés Bautista}{937/21}{andycia802@gmail.com}
\integrante{Perez Lanzillotta, Santiago}{586/16}{santi.perezl@hotmail.com}

\begin{document}

\maketitle



\section{Especificaci'on}

\subsection{redistribucionDeLosFrutos}

%  \  Ejercicio 1

\begin{proc}{redistribucionDeLosFrutos}{\In recursos : \TLista{\float}, \In cooperan : \TLista{\bool}}{\TLista{\float}}
	\requiere{(\forall \text{ recurso } \in \text{ recursos } \implicaLuego {recurso \geq 0) \land |recursos| = |cooperan|}
}
	\asegura {\\
 |res| = |cooperan| \yLuego \\ 
 \paraTodo[unalinea]{i}{\ent} {(0 \leq i < |res|)  \implicaLuego \\ (\IfThenElse{cooperan[i] = True}{res[i] = \frac{fondo(recursos,  cooperan)}{|cooperan|}}{res[i] = recursos[i] + \frac{fondo(recursos,cooperan)}{|cooperan|}} }) \\
 }

\end{proc}
\vspace{0.50cm}

\aux{fondo}{in recursos : {\TLista{\ent}}, in cooperan : {\TLista{\bool}}}{\ent} {
\sum\limits_{i=0} \limits^{|recursos|-1} 
({\IfThenElse{cooperan[i] \hspace{0.15cm} = \hspace{0.15cm} True}{recursos[i]}{0}})}
    
 % \   {\ifthenelse{\equal{cooperan[i]}{True}}{recursos[i]}{0}}}

%  \   {if cooperan[i] = True then 1 else 0 fi}}



\vspace{1.2cm}

\subsection{trayectoriaDeLosFrutosIndividualesALargoPlazo}

%  \  Ejercicio 2

\begin{proc}{trayectoriaDeLosFrutosIndividualesALargoPlazo}{\Inout trayectorias : \TLista{\TLista{\float}}, \In cooperan : \TLista{\bool}, \In apuestas : \TLista{\TLista{\float}}, \In pagos : \TLista{\TLista{\float}}, \In eventos : \TLista{\TLista{\bool}}}{\TLista{\TLista{\float}}}
% requiere entradasValidas (misma cantidad de jugadores en todas los parámetros de entrada)
    \requiere{|trayectorias_{0}| = |trayectoria|  \yLuego \\ \paraTodo[unalinea]{individuo}{\nat}{0 \leq individuo < |trayectorias| \implicaLuego (|trayectorias[individuo]| > 0 \yLuego \\ |trayectorias[individuo]| = |apuestas| + 1 = |pagos| + 1 = |eventos| + 1) \yLuego \\ (|cooperan| = |trayectorias| = |apuestas[individuo]| = |pagos[individuo]| = |eventos[individuo]|)}}
    \asegura{\\
    \paraTodo[unalinea]{individuo}{\nat}{0 \leq individuo < |trayectorias|  \implicaLuego |trayectorias[individuo]| = |trayectorias_{0}[individuo]| + 1  \land \paraTodo[unalinea]{\omega}{\ent}{0 \leq \omega < |matrizTraspuesta(trayectorias)| - 1 \implicaLuego \newline matrizTraspuesta[\omega + 1] = \\ redistribucionDeLosFrutos(ganancias(matrizTraspuesta(trayectoria)[|\omega|], matrizTraspuesta(apuestas)[|\omega|], \\ matrizTraspuesta(pagos)[|\omega|], matrizTraspuesta(eventos)[|\omega|]), matrizTraspuesta(cooperan)[|\omega|])} \yLuego \\ matrizTraspuesta(trayectorias)[\omega] = matrizTraspuesta(trayectorias_{0})[\omega]}\\
    }

\end{proc}
\vspace{0.50cm}

\aux{ganancias}{\In ultimasTrayectorias : \TLista{\TLista{\float}}, \In ultimasApuestas : \TLista{\TLista{\float}}, \In ultimosPagos : \TLista{\TLista{\float}}, \In ultimosEventos : \TLista{\TLista{\nat}}}{\TLista{\TLista{\float}}}{\newline{\paraTodo[unalinea]{individuo}{\nat}{0 \leq individuo < |ultimasApuestas| \implicaLuego {$res[i]$ = \\ \IfThenElse {ultimasApuestas[individuo] = ultimosEventos[individuo]}{ultimosPagos[individuo]}{0}}}}}

\vspace{0.50cm}

\aux{matrizTraspuesta}{\Inout matriz : \TLista{\TLista{\float}}}{\TLista{\float}}{\\ \paraTodo[unalinea]{fila}{matriz}{\paraTodo[unalinea]{columna}{matriz}{matriz[fila][columna] = old(matriz[columna] [fila])}}}


\newpage

\subsection{trayectoriaExtrañaEscalera}


%  \  Ejercicio 3

\begin{proc}{trayectoriaExtrañaEscalera}{\In trayectoria : \TLista{\float}}{\bool}
	\requiere{|trayectoria| > 0}
	\asegura{\\
 res = true \Leftrightarrow \\
 (|trayectoria| = 1)  \vee  \\
(|trayectoria| = 2 \yLuego (trayectoria[0] \neq trayectoria[1] )) \vee \\
(|trayectoria| >= 3 \yLuego \\
((ContadorMayoresExtremoMedio(trayectoria) + \\
ContadorMayoresExtremoIzquierdo(trayectoria) + \\
ContadorMayoresExtremoDerecho(trayectoria)) = 1))
 
 }
\vspace{0.8cm}
 \end{proc}
 \aux{ContadorMayoresExtremoMedio}{\In trayectoria : {\TLista{\float}}}{\ent} {\\
\sum\limits_{i=1} \limits^{|trayectoria|-2} 
({\IfThenElse{(trayectoria[i-1] < trayectoria[i]) \land (trayectoria[i]>trayectoria[i+1]) \hspace{0.15cm} = \hspace{0.15cm} True}{1}{0}})}
\vspace{0.8cm}
\aux{ContadorMayoresExtremoIzquierdo}{\In trayectoria : {\TLista{\float}}}{\ent} {\\
{\IfThenElse{(trayectoria[0] > trayectoria[1]) \hspace{0.15cm} = \hspace{0.15cm} True}{1}{0}})}

\vspace{0.8cm}
\aux{ContadorMayoresExtremoDerecho}{\In trayectoria : {\TLista{\float}}}{\ent} {\\
{\IfThenElse{(trayectoria[|trayectoria|-1] > trayectoria[|trayectoria|-2]) \hspace{0.15cm} = \hspace{0.15cm} True}{1}{0}})}

\vspace{1.2cm}

\subsection{individuoDecideSiCooperarONo}

%  \  Ejercicio 4

\begin{proc}{individuoDecideSiCooperarONo}{\In individuo : \nat, \In recursos : \TLista{\float}, \Inout cooperan : \TLista{\bool}, \In apuestas : \TLista{\TLista{\float}}, \In pagos : \TLista{\TLista{\float}}, \In eventos : \TLista{\TLista{\nat}}}{\TLista{\bool}}
	\requiere{0 < individuo < |cooperan| \yLuego |apuestas| = |pagos| = |eventos| \yLuego \paraTodo[unalinea]{jugadores}{\nat}{|cooperan| = |recursos| = |apuestas[jugadores]| = |pagos[jugadores]| = |eventos[jugadores]|} \yLuego 0 \leq individuo < |recursos| \yLuego \paraTodo[unalinea]{recurso}{recursos}{recurso \geq 0}}
	\asegura{cooperan[individuo] = \IfThenElse{trayectoriaDeLosFrutosIndividualesALargoPlazo(\\ \TLista{recursos}, setAt(cooperan, individuo, True), apuestas, pagos, eventos)[individuo[|eventos|]] \geq \\ trayectoriaDeLosFrutosIndividualesALargoPlazo(\\ \TLista{recursos}, setAt(cooperan, individuo, False), apuestas, pagos, eventos)[individuo[|eventos|]]}{True}{False}}

%   Deprecated:   \  cooperan[individuo] = \IfThenElse{\existe{j}{\ent}{0 \leq j < |cooperan| \implicaLuego (cooperan[j] = False) \land apuestas[j] > 0 \land pagos[j] > 0 \land }}{False}{True}} \yLuego \paraTodo[unalinea]{i}{cooperan}{i \neq individuo 

\end{proc}


\vspace{1.2cm}



\subsection{individuoActualizaApuesta}

%  \  Ejercicio 5

% \ llama al proc del ejercicio 2, y asegura que el resultado de este ejercicio maximiza la siguiente expresión de la siguiente forma: res \geq trayectorias[individuo[|trayectoria[individuo]|]]

%     \    IDEA:

\begin{proc}{individuoActualizaApuesta}{\In individuo : \nat, \In recursos : \TLista{\float}, \In cooperan : \TLista{\bool}, \Inout apuestas : \TLista{\TLista{\float}}, \In pagos : \TLista{\TLista{\float}}, \In eventos : \TLista{\TLista{\bool}}}{\TLista{\TLista{\float}}}
    \requiere{0 < individuo < |cooperan|}
    \asegura{\paraTodo[unalinea]{a}{apuestas[individuo]}{\paraTodo[unalinea]{b}{\nat}{(0 \leq a, b \leq 1) \implicaLuego \\ (trayectoriaDeLosFrutosIndividualesALargoPlazo(\\ \TLista{recursos}, cooperan, setAt(apuestas, individuo, a), pagos, eventos)[individuo[|eventos|]] \geq \\ trayectoriaDeLosFrutosIndividualesALargoPlazo(\\ \TLista{recursos}, cooperan, setAt(apuestas, individuo, b), pagos, eventos)[individuo[|eventos|]])}}}

\end{proc}

\end{document}
